\subsection{Architectural Overview}
The system is built on a client–server architecture, separating concerns between a \textbf{React frontend} and an \textbf{Express backend}. This modular design ensures that each component can operate, scale, and be maintained independently. The backend manages business logic and database interactions, while the frontend is responsible for presenting information and capturing user interaction. Communication between the two layers occurs exclusively via RESTful API calls, maintaining a strict separation of concerns and minimizing coupling.

\subsection{Evaluation of Key Architectural Aspects}

\begin{table}[H]
\centering
\begin{tabularx}{\linewidth}{|l|X|}
\hline
\textbf{Aspect} & \textbf{Evaluation} \\
\hline
Logical Separation & The frontend and backend are completely decoupled; the frontend has no direct access to backend logic or databases. \\
\hline
Communication (API) & All interactions occur via RESTful APIs, preserving a stateless design and enabling consistent decoupling. \\
\hline
Security & Authentication and authorization are handled server-side. The frontend never handles sensitive logic or has direct access to user data. \\
\hline
Responsibility Split & The client manages rendering and local UI state; the server handles data validation, persistence, and all core processing. \\
\hline
\end{tabularx}
\caption{Architectural Aspect Evaluation}
\end{table}

\subsection{Strengths}
\begin{itemize}
    \item \textbf{Scalability}: The backend can be scaled independently (e.g., through load balancing), while the frontend is lightweight and can be served statically.
    \item \textbf{Maintainability}: Clear separation of concerns allows UI changes without backend modifications and vice versa, improving long-term maintainability.
    \item \textbf{Flexibility}: The system architecture supports multiple types of clients using the same backend API.
    \item \textbf{Security}: By centralizing sensitive operations on the server, the system reduces exposure to potential vulnerabilities.
\end{itemize}

\subsection{Limitations}
\begin{itemize}
    \item \textbf{Database Integration}: Local databases are currently in use, limiting centralization and real-time synchronization between users.
    \item \textbf{Feature Scope}: Some intended features remain unimplemented due to time constraints; future iterations could further enhance system capability.
    \item \textbf{User Interface}: Functionality was prioritized over visual design, resulting in a utilitarian UI that could benefit from aesthetic improvements.
\end{itemize}

\subsection{Alignment with Project Goals}

\begin{table}[H]
\centering
\noindent\begin{tabularx}{\linewidth}{|l|X|}
\hline
\textbf{Goal} & \textbf{Implementation} \\
\hline
Sustainable Study Habits & The server enforces daily point caps to promote consistent, healthy engagement patterns. \\
\hline
Gamification and Rewards & The client displays avatars and themes based on reward data retrieved from the backend. \\
\hline
Scalability & The architecture supports increasing user loads without compromising client performance. \\
\hline
Educational Feedback & The server aggregates and processes user performance data; the client renders this feedback to users. \\
\hline
\end{tabularx}
\caption{Project Goals and Architectural Alignment}
\end{table}
