\subsection{Studying: Cramming vs. Spaced Repetition}
\label{sec:cramming-vs-spaced-repetition}
In the context of studying, cramming refers to the practice of trying to learn a large amount of information in a short period of time, often just before an exam. This method is often associated with high levels of stress and anxiety, and while it may lead to short-term retention of information, it is generally not effective for long-term learning.

For a more effective approach to studying, spaced repetition is recommended. This method involves reviewing material at increasing intervals over time, which has been shown to improve retention and understanding of the material. Spaced repetition takes advantage of the brain's natural forgetting curve, allowing learners to reinforce their memory just before they are likely to forget the information.\cite{cepeda2006distributed}

In the case of this application, the goal is to incentivize students to study in a spaced manner by allowing them to take short quizzes on the material they are learning. While they are taking the quizzes, they can earn points that can be redeemed for rewards.\cite{liu2022immediate} This approach not only encourages students to study more effectively but also provides them with immediate feedback on their understanding of the material.\cite{staddon2003operant} To further incentivize students to come back to the quizzes, points are capped at a daily limit set by their instructor. This means that students will need to return to the quizzes on multiple days in order to earn the maximum number of points possible.

\subsection{Current Solutions}
\label{sec:current-solutions}

There are several existing solutions that aim to help students study more effectively. Some of these solutions include:

\begin{itemize}
  \item Quizlet: A popular online platform that allows students to create and share flashcards, quizzes, and study games. Quizlet also offers a spaced repetition feature called "Learn" that helps students retain information over time.
  \item Kahoot!: A game-based learning platform that allows teachers to create quizzes and games for their students. Kahoot! is designed to be engaging and interactive, making it a popular choice for classroom learning.
  \item Anki: A flashcard app that uses spaced repetition to help users learn and retain information. Anki is highly customizable and allows users to create their own flashcards or download pre-made decks.
\end{itemize}

While these solutions have their strengths, they lack the "incentive" aspect that this application aims to provide. By allowing students to earn actual rewards for their studying, this application seeks to create a more engaging and motivating learning experience.\cite{smithsonian2016benefits} Additionally, the ability to set daily point caps encourages students to return to the quizzes on multiple days, further reinforcing their learning through spaced repetition.