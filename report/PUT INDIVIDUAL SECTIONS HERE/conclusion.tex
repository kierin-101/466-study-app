The app successfully promotes evidence-backed studying methods, such as spaced repetition, through the implementation of daily points caps. Each class has an individual cap set by the teacher at creation, allowing independence between classes and preventing students from neglecting certain classes if they've already exhausted their daily incentives. Reinforcement by earning points builds daily studying habits. Teachers can take an active role in the preparation of their students by providing tailored quizzes for their classes that target specific information deemed crucial for understanding, as well as set target scores for the created quizzes.

The primary goal that could not be completed during course time is providing analytical tools for teachers to see the performance of their students on quizzes and individual questions. These statistics may indicate widespread successes or misunderstandings that help adjust teaching methods. Records of user answers for a quiz, including on individual attempts, are already stored in the database upon quiz submission, so this would be relatively simple to implement as part of the teacher view. Pairing with this, new functionality to revise quizzes after creating them would be useful.

Beyond this, the current system is limited in that it relies on local MSSQL databases to store data. Having a single shared database would be essential if this app were released to a real user setting. Additionally, due to time constraints, image assets for the rewards are currently kept on the client side rather than pulled from the server, meaning new rewards require updates to both the server and the client’s static files. Server-side BLOB storage of these assets would be ideal to remove this dependency. As part of these server upgrades, additional input validation on client-side form fields would be required to better protect data, such as against injection attacks.

Potential expansions to functionality include enabling classes to have multiple teachers, perhaps through explicit invitation from an existing teacher rather than use of a join code, to prevent undue changes to the preexisting system architecture. Lastly, allowing teachers to reuse quizzes from other classes or adding public quizzes for all students to access would be a useful tool that matches services provided by similar studying platforms.

Finally, improving the user interface and user experience would elevate the app significantly. Functionality was prioritized over aesthetics, so the current product is very simple in appearance, at the cost of its long-term appeal. Styling the pages and providing additional visual feedback for things like button clicks would add interest and help keep users engaged. Extending the role of the rewards, such as allowing themes to influence the visual design of pages, would add to the cohesion of the app and increase the appeal of the rewards.

Overall, though there are a multitude of ways the app could continue to be developed and improved into a more complete product, the core functionality has been finished in an adaptable and architecturally sound state. Thanks to the early identification of a favorable architecture for the system in the form of the client-server style, and subsequent reframing of our requirements in this context, direction for the app was well-defined from the early stages. Subsequent stages saw us consulting and revisiting this architecture as the features were developed, allowing smooth negotiation of which layers would handle the minutiae of the implementation. Future work is expected to work smoothly within the existing system thanks to this foundation, attesting to the crucial role architecture plays in the long-term maintenance and evolution of a software system.
